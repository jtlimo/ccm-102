\documentclass[12pt]{article}

\usepackage{sbc-template}
\usepackage{graphicx,url}
\usepackage[utf8]{inputenc}
\usepackage[brazil]{babel}
\usepackage{comment}
\usepackage{lipsum}

\sloppy

\title{Análise comparativa de métricas de avaliação\\
em modelos de aprendizado de máquina na detecção de\\
ataques de spoofing de GPS
}

\author{Ana Carla Rodrigues\inst{1}, Jessica Fileto\inst{1}, Jhossett Longobardi\inst{1}}

\address{
  Centro de Matemática, Computação e Cognição\\
  Universidade Federal do ABC (UFABC)\\
  Av. dos Estados, 5001 -- 09210-580 -- Santo André -- SP -- Brasil
  \email{\{carla.rodrigues,jessica.fileto,stalinn.longobardi\}@ufabc.edu.br}
}

\begin{document} 

\maketitle

\begin{abstract}
Unmanned aerial systems (UAS), commonly known as drones, are transformative 
platforms that revolutionize industries around the world by offering highly 
flexible solutions with reduced operational risks. Although the Global 
Positioning System (GPS) enables precise navigation, tracking, and autonomous 
flight for UAS, these systems remain vulnerable to GPS spoofing attacks, which 
compromise data integrity and underscore the critical role of accurate data 
analysis in decision-making. Artificial intelligence can mitigate such attacks 
through sophisticated techniques, reducing false alarms and providing reliable 
threat classification via machine learning. This project proposes a comparative 
analysis of evaluation metrics in three machine learning models: Random Forest, 
Bagged Decision Trees and Extremely Randomized Trees, to identify the optimal 
approach to balance the accuracy of spoofing detection and the feasibility of 
model implementation.
\end{abstract}
     
\begin{resumo} 
O Veículo Aéreo Não Tripulado (VANT), comumente conhecido como drone, é uma
plataforma que está revolucionando indústrias ao redor do mundo, oferecendo 
soluções de alta flexibilidade e riscos reduzidos. O Global Positioning System 
(GPS) fornece ao VANT navegação precisa, rastreamento e vôo autônomo. Os VANTs 
podem sofrer ataques de spoofing nas coordenadas de GPS, comprometendo a 
segurança da informação e demonstrando que a precisão na análise dos dados são 
cruciais na tomada de decisões. A inteligência artificial pode ser uma aliada 
na detecção destes ataques através de técnicas sofisticadas, reduzindo a 
quantidade de falsos alarmes e fornecendo informações confiáveis por meio do 
aprendizado de máquina que exerce papel fundamental na classificação destes 
ataques. A proposta deste projeto é realizar uma análise comparativa das 
métricas de avaliação dos seguintes modelos de aprendizado de máquina: Random 
Forest, Bagged Decision Trees e Extremely Randomized Trees, visando identificar 
aquele que melhor equilibre a detecção correta de spoofing e viabilidade de 
implementação do modelo.
\end{resumo}

\section{Introdução}
% Aqui sem divisão as nossas partes, 
% Contextualização Geral & Revisão de Literatura
% Justificativa & Objetivos
% Delimitação e Escopo & Organização do Trabalho

%============================= Ana ============================================%



%============================= Jessica ========================================%

\section{Justificativa}

Os veículos aéreos não tripulados (VANTs), popularmente conhecidos como drones são amplamente utilizados, tanto no âmbito civil, quanto no militar. \cite{capitanFrameworkHandleThreats2019}

Alguns exemplos de uso são: gerenciamento de desastres, vigilância aérea, fotografia aérea de rastreamento, pesquisa e resgate, monitoramento de gado, dentre outros. \cite{titounaLightweightSecurityTechnique2021}

O GPS é o que torna os VANTs autônomos e capazes de concluir de maneira automatizada as diferentes tarefas a que são atribuídos. Estas informações são cruciais para seu desempenho e locomoção correta. A comunicação dos VANTs civis, diferente dos militares, utiliza comunicação desencriptada e não autenticada. Como este dispositivo depende primariamente de dados de localização de GPS, ele fica extremamente vulnerável a diversos tipos de ataques, principalmente os conhecidos como \textit{Spoofing de GPS}. \cite{srinivasansGPSSpoofingDetection2023,titounaLightweightSecurityTechnique2021}

\begin{quote}
  Como resultado, garantir a segurança cibernética [\dots] se tornou uma preocupação crucial, abrangendo falhas potenciais do sistema e ataques externos em seus componentes. Isso destaca a necessidade de extensas pesquisas na segurança cibernética do VANT, que se concentram na detecção e prevenção de ataques e falhas.  \cite[tradução nossa]{elalamiDroneDefGANtGenerativeAIBased2024}
\end{quote}




\section{Objetivos}

%=========================== Stalinn ==========================================%

%=========================== Stalinn ==========================================%

%Delimitação e Escopo
O \textit{dataset} foi tomado da base de dados do \textit{Mendeley Data} 
\cite{aissou2022dataset}; esses dados foram gerados de sinais GPS autenticas e 
contem aproximadamente 500,000 dados. Então, os resultados do trabalho ficam 
limitados à esse \textit{dataset}.

Nas proximas seções se estabelece a execução do trabalho. Na seção 2, fazemos 
nossa proposta de mudança ao trabalho de \cite{Aissou2021}, discutindo a 
metodologia e modelos de aprendizado de maquina a
serem implementados. Logo, os resultados e 
discusões são apresentados na seção 3. Finalmente, nossas conclusões sobre o 
trabalho são expostas na seção 4.

\section{Proposta de mudança (Metodologia)} % A outra seção que temos que armar

\lipsum[1-4]



\section{Proposta de mudança (Metodologia)} % A outra seção que temos que armar



%%%%%%%%%%%%%%%%%%%%%%%%%%%%%%%%%%%%%%%%%%%%%%%%%%%%%%%%%%%%%%%%%%%%%%%%%%%%%%%%
%%%%%%%%%%%%%%%%%%%%%% Instructions %%%%%%%%%%%%%%%%%%%%%%%%%%%%%%%%%%%%%%%%%%%%

\begin{comment}

\section{General Information}

All full papers and posters (short papers) submitted to some SBC conference,
including any supporting documents, should be written in English or in
Portuguese. The format paper should be A4 with single column, 3.5 cm for upper
margin, 2.5 cm for bottom margin and 3.0 cm for lateral margins, without
headers or footers. The main font must be Times, 12 point nominal size, with 6
points of space before each paragraph. Page numbers must be suppressed.

Full papers must respect the page limits defined by the conference.
Conferences that publish just abstracts ask for \textbf{one}-page texts.

\section{First Page} \label{sec:firstpage}

The first page must display the paper title, the name and address of the
authors, the abstract in English and ``resumo'' in Portuguese (``resumos'' are
required only for papers written in Portuguese). The title must be centered
over the whole page, in 16 point boldface font and with 12 points of space
before itself. Author names must be centered in 12 point font, bold, all of
them disposed in the same line, separated by commas and with 12 points of
space after the title. Addresses must be centered in 12 point font, also with
12 points of space after the authors' names. E-mail addresses should be
written using font Courier New, 10 point nominal size, with 6 points of space
before and 6 points of space after.

The abstract and ``resumo'' (if is the case) must be in 12 point Times font,
indented 0.8cm on both sides. The word \textbf{Abstract} and \textbf{Resumo},
should be written in boldface and must precede the text.

\section{CD-ROMs and Printed Proceedings}

In some conferences, the papers are published on CD-ROM while only the
abstract is published in the printed Proceedings. In this case, authors are
invited to prepare two final versions of the paper. One, complete, to be
published on the CD and the other, containing only the first page, with
abstract and ``resumo'' (for papers in Portuguese).

\section{Sections and Paragraphs}

Section titles must be in boldface, 13pt, flush left. There should be an extra
12 pt of space before each title. Section numbering is optional. The first
paragraph of each section should not be indented, while the first lines of
subsequent paragraphs should be indented by 1.27 cm.

\subsection{Subsections}

The subsection titles must be in boldface, 12pt, flush left.

\section{Figures and Captions}\label{sec:figs}


Figure and table captions should be centered if less than one line
(Figure~\ref{fig:exampleFig1}), otherwise justified and indented by 0.8cm on
both margins, as shown in Figure~\ref{fig:exampleFig2}. The caption font must
be Helvetica, 10 point, boldface, with 6 points of space before and after each
caption.

\begin{figure}[ht]
\centering
\includegraphics[width=.5\textwidth]{fig1.jpg}
\caption{A typical figure}
\label{fig:exampleFig1}
\end{figure}

\begin{figure}[ht]
\centering
\includegraphics[width=.3\textwidth]{fig2.jpg}
\caption{This figure is an example of a figure caption taking more than one
  line and justified considering margins mentioned in Section~\ref{sec:figs}.}
\label{fig:exampleFig2}
\end{figure}

In tables, try to avoid the use of colored or shaded backgrounds, and avoid
thick, doubled, or unnecessary framing lines. When reporting empirical data,
do not use more decimal digits than warranted by their precision and
reproducibility. Table caption must be placed before the table (see Table 1)
and the font used must also be Helvetica, 10 point, boldface, with 6 points of
space before and after each caption.

\begin{table}[ht]
\centering
\caption{Variables to be considered on the evaluation of interaction
  techniques}
\label{tab:exTable1}
\includegraphics[width=.7\textwidth]{table.jpg}
\end{table}

\section{Images}

All images and illustrations should be in black-and-white, or gray tones,
excepting for the papers that will be electronically available (on CD-ROMs,
internet, etc.). The image resolution on paper should be about 600 dpi for
black-and-white images, and 150-300 dpi for grayscale images.  Do not include
images with excessive resolution, as they may take hours to print, without any
visible difference in the result. 



Bibliographic references must be unambiguous and uniform.  We recommend giving
the author names references in brackets, e.g. 

The references must be listed using 12 point font size, with 6 points of space
before each reference. The first line of each reference should not be
indented, while the subsequent should be indented by 0.5 cm.

\end{comment}

%%%%%%%%%%%%%%%%%%%%%%%%%%%%%% End %%%%%%%%%%%%%%%%%%%%%%%%%%%%%%%%%%%%%%%%%%%%%
%%%%%%%%%%%%%%%%%%%%%%%%%%%%%%%%%%%%%%%%%%%%%%%%%%%%%%%%%%%%%%%%%%%%%%%%%%%%%%%%%

\bibliographystyle{sbc}
\bibliography{references}


\end{document}
