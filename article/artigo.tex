\documentclass[12pt]{article}

\usepackage{sbc-template}
\usepackage{graphicx,url}
\usepackage[utf8]{inputenc}
\usepackage[brazil]{babel}
\usepackage{comment}
\usepackage{lipsum}
\usepackage{csquotes}
\usepackage{url}
\usepackage[
    style=authoryear,
    citestyle=authoryear,
    backend=biber
]{biblatex}

\addbibresource{references.bib}

\DeclareFieldFormat{labelyear}{#1}
\renewcommand*{\nameyeardelim}{\addspace}
\renewbibmacro*{cite}{%
  \printtext[brackets]{%
    \printnames{labelname}%
    \setunit{\nameyeardelim}%
    \printfield{labelyear}%
  }}

\urlstyle{same} 

\sloppy

\title{Análise comparativa de métricas de avaliação\\
em modelos de aprendizado de máquina na detecção de\\
ataques de spoofing de GPS
}

\author{Ana Carla Rodrigues\inst{1}, Jessica Fileto\inst{1}}

\address{
  Centro de Matemática, Computação e Cognição\\
  Universidade Federal do ABC (UFABC)\\
  Av. dos Estados, 5001 -- 09210-580 -- Santo André -- SP -- Brasil
  \email{\{carla.rodrigues,jessica.fileto\}@ufabc.edu.br}
}

\begin{document}

\maketitle
     
\begin{resumo} 
O Veículo Aéreo Não Tripulado (VANT), comumente conhecido como drone, é uma
plataforma que está revolucionando indústrias ao redor do mundo, oferecendo 
soluções de alta flexibilidade e riscos reduzidos. O Global Positioning System 
(GPS) fornece ao VANT navegação precisa, rastreamento e vôo autônomo. Os VANTs 
podem sofrer ataques de spoofing nas coordenadas de GPS, comprometendo a 
segurança da informação e demonstrando que a precisão na análise dos dados são 
cruciais na tomada de decisões. A inteligência artificial pode ser uma aliada 
na detecção destes ataques através de técnicas sofisticadas, reduzindo a 
quantidade de falsos alarmes e fornecendo informações confiáveis por meio do 
aprendizado de máquina que exerce papel fundamental na classificação destes 
ataques. A proposta deste projeto é realizar uma análise comparativa das 
métricas de avaliação dos seguintes modelos de aprendizado de máquina:
Random Forest, Naive Bayes, K-NN, Gaussian Naive Bayes e Extremely Randomized Trees,
visando identificar aquele que melhor equilibre a detecção correta
de spoofing e viabilidade de implementação do modelo.
\end{resumo}

\section{Introdução}

Os veículos aéreos não tripulados (VANTS), também conhecidos como drone,
são um tipo de tecnologia fundamental tanto em áreas civis quanto militares.
Seu uso reduz a exposição humana a tarefas repetitivas, de longa duração
e em alguns casos, de alto risco \cite{dialogos}.
Desempenham papel importante nas seguintes áreas: gerenciamento de desastres,
vigilância aérea, fotografia aérea de rastreamento, pesquisa e resgate,
monitoramento de gado, dentre outros. \cite{titounaLightweightSecurityTechnique2021} 

Por serem operados de maneira autônoma e remota suas coordenadas são essenciais
para o sucesso de suas missões, para isso são usados sinais
de Sistema Global de Navegação por Satélite (GNSS),
sendo o mais conhecido o \textit{Global Positioning System} (GPS),
esses sinais podem ser criptografados ou não \cite {lester}.
Sendo os não criptografados mais suscetíveis a ataques conhecidos como
\textit{Spoofing} de GPS \cite{srinivasansGPSSpoofingDetection2023}. 

Para garantir um vôo seguro do VANT é essencial ter medidas
para detecção dos ataques de \textit{Spoofing de GPS}.
A identificação precisa desses ataques é extremamente importante,
pois técnicas avançadas de \textit{spoofing}
podem causar disrupção no funcionamento do VANT,
portanto são necessários métodos robustos e adaptativos
\cite{isleyenGPSSpoofingDetection2024}.
Nesse contexto algoritmos de aprendizado de máquina tem se mostrado promissores,
pois são capazes de analisar grandes volumetrias de dados,
identificando padrões e anomalias.
Tais abordagens oferecem uma resposta robusta e adaptativa
a essas ameaças contribuindo para as operações dos VANTs
\cite{isleyenGPSSpoofingDetection2024}.

\section{Trabalhos relacionados}

Diversos estudos recentes tem utilizado
essa abordagem para detecção de \textit{spoofing} de GPS.
Podemos mencionar algumas abordagens utilizadas:

\begin{itemize}
    \item \textbf{Redes neurais profundas (DNN):}
     Extração de padrões complexos em séries temporais dos sinais GNSS.
     \cite{isleyenGPSSpoofingDetection2024}

    \item \textbf{Máquinas de vetores de suporte (SVM):}

     Abordagem teve como objetivo a identificação
     de discrepâncias entre as posições determinadas pelos sinais de GPS
     e aquelas medidas pelo sistema de navegação inercial (INS).
     Especificamente, o modelo SVM foi treinado através dos erros obtidos
     das diferenças posicionais,
     permitindo que o sistema detecte inconsistências que poderiam
     indicar um ataque de \textit{spoofing} de GPS.
     \cite{panice2017}
    
    \item \textbf{Redes neurais convolucionais (CNN):} Em comparação
    aos modelos tradicionais de aprendizagem de máquina,
    os de aprendizagem profunda obtiveram alta acurácia na detecção dos
    ataques de \textit{spoofing}.
    A grande vantagem desses modelos baseados em aprendizagem profunda,
    é pelo fato de aprenderem automaticamente a extraírem as \textit{features}
    sem precisarem de intervenção humana. Além de se adaptarem melhor
    a \textit{datasets} mais complexos. \cite{cnn2023}

    
    \item \textbf{Aprendizagem supervisionada (Baseado em Árvores):}
    \textcite{Aissou2021} utilizou os seguintes modelos baseados
     em árvores: Random Forest, Gradient Boost, XGBoost e LightGBM
     para fazer um comparativo de qual seria melhor na detecção dos ataques
     de \textit{spoofing} de GPS.
     Sendo que o XGBoost obteve a melhor acurácia (95,52\%).

    \item \textbf{IA Generativa:}  Abordagem que em comparação com outros
    modelos de aprendizagem de máquina, se destacou pela eficácia
    na deteção dos ataques de \textit{spoofing} e \textit{jamming}.
    \cite{elalamiDroneDefGANtGenerativeAIBased2024}
    
\end{itemize}

\section{Delimitação e Escopo}
O \textit{dataset} que será utilizado como base de dados é o \textit{Mendeley Data} 
\cite{aissou2022dataset}, esses dados foram gerados de sinais GPS autênticos e 
contém aproximadamente 500,000 dados, no artigo escolhido de \cite{Aissou2021} não 
possui uma citação do \textit{dataset} usado para o treinamento dos modelos, 
então para as análises decorrentes deste artigo será utilizada essa fonte de dados 
simplificada e disponibilizada pelos autores do artigo. 
De acordo com o artigo de \cite{Aissou2021}, os ataques do tipo
\textit{spoofing} podem ser classificados em: simples,
intermediários e sofisticados.

Nas próximas seções se estabelecerá a execução do trabalho. Na seção 4, faremos 
nossa proposta de mudança ao trabalho de \cite{Aissou2021}, discutindo a 
metodologia e modelos de aprendizado de máquina a
serem implementados. Logo, os resultados e 
discussões serão apresentados na seção 5.
Finalmente, nossas conclusões sobre o trabalho serão expostos na seção 6.

\section{Metodologia (Proposta de mudança)}


A metodologia adotada e os passos seguidos
foram inspirados no artigo de \cite{Aissou2021}, 
que utilizou a implementações de algoritmos como: 
\textit{Random Forest}, \textit{Gradient Boost}, \textit{XGBoost}, e 
\textit{LightGBM} com o objetivo de explorar os dados obtidos.

A proposta de mudança deste artigo é avaliar outros modelos
para a classificação de possíveis ataques de \textit{spoofing} 
a sinais de GPS, com enfoque em algoritmos que não são baseados em árvore, sendo estes:
(i) \textit{Naive Bayes}, (ii) \textit{K-NN} e (iii) \textit{Gaussian Naive Bayes}.

Além da reimplementação do algoritmo (iv) \textit{Random Forest}
que foi utilizado no artigo \cite{Aissou2021} e obteve o pior desempenho no comparativo.
Com base no desempenho do \textit{Random Forest}, será feita a implementação do algoritmo 
(v) \textit{Extremely randomized trees}, um modelo melhorado do \textit{Random Forest},
que se diferem pelo fato de no \textit{Extremely randomized trees}
não existir a fase de \textit{bagging} e no momento da separação dos nós,
esta escolha é feita aleatoriamente. \cite{geurtsExtremelyRandomizedTrees2006}

Serão analisados a acurácia, 
precisão e a quantidade de falsos negativos. 
Para isso a base será dividida em porções
diferentes para testes e o treinamento do modelo,
sendo 70\% para treino e 30\% para testes como no artigo de \cite{Aissou2021}. 

Será aplicada a técnica de pré processamento
\textit{Principal Component Analysis} (PCA) para melhorar
a precisão dos resultados finais, ela consiste em reduzir as
dimensões do dataset para deixar apenas as informações mais importantes,
removendo assim as redundâncias \cite{IBM02025}. 
Esse tipo de processamento ajuda algoritmos de aprendizagem de máquina,
simplificando o processo de reconhecimento dos dados, essa técnica é
diferente da usada no artigo de \cite{Aissou2021}
que se chama \textit{Spearman Correlation Coefficient}.

%\subsection{Algoritmos}

\printbibliography

\end{document}
