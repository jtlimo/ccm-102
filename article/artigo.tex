\documentclass[12pt]{article}

\usepackage{sbc-template}
\usepackage{graphicx,url}
\usepackage[utf8]{inputenc}
\usepackage[brazil]{babel}
\usepackage{comment}
\usepackage{lipsum}
\usepackage{csquotes}
\usepackage{url}

\urlstyle{same} 

\sloppy

\title{Análise comparativa de métricas de avaliação\\
em modelos de aprendizado de máquina na detecção de\\
ataques de spoofing de GPS
}

\author{Ana Carla Rodrigues\inst{1}, Jessica Fileto\inst{1}}

\address{
  Centro de Matemática, Computação e Cognição\\
  Universidade Federal do ABC (UFABC)\\
  Av. dos Estados, 5001 -- 09210-580 -- Santo André -- SP -- Brasil
  \email{\{carla.rodrigues,jessica.fileto\}@ufabc.edu.br}
}

\begin{document}

\maketitle
     
\begin{resumo} 
O Veículo Aéreo Não Tripulado (VANT), comumente conhecido como drone, é uma
plataforma que está revolucionando indústrias ao redor do mundo, oferecendo 
soluções de alta flexibilidade e riscos reduzidos. O Global Positioning System 
(GPS) fornece ao VANT navegação precisa, rastreamento e vôo autônomo. Os VANTs 
podem sofrer ataques de spoofing nas coordenadas de GPS, comprometendo a 
segurança da informação e demonstrando que a precisão na análise dos dados são 
cruciais na tomada de decisões. A inteligência artificial pode ser uma aliada 
na detecção destes ataques através de técnicas sofisticadas, reduzindo a 
quantidade de falsos alarmes e fornecendo informações confiáveis por meio do 
aprendizado de máquina que exerce papel fundamental na classificação destes 
ataques. A proposta deste projeto é realizar uma análise comparativa das 
métricas de avaliação dos seguintes modelos de aprendizado de máquina: Random 
Forest, Bagged Decision Trees e Extremely Randomized Trees, visando identificar 
aquele que melhor equilibre a detecção correta de spoofing e viabilidade de 
implementação do modelo.
\end{resumo}

\section{Introdução}
% Aqui sem divisão as nossas partes, 
% Contextualização Geral & Revisão de Literatura
% Justificativa & Objetivos
% Delimitação e Escopo & Organização do Trabalho

%============================= Ana e Jessica ============================================%
Os veículos aéreos não tripulados (VANTS), também conhecidos como drone,
é um tipo de tecnologia fundamental tanto em áreas civis quanto militares.
O uso de drones reduzem a exposição humana a tarefas repetitivas, de longa duração e em alguns casos, de alto risco \cite{dialogos}.
Desempenham papel importante nas seguintes áreas: gerenciamento de desastres, vigilância aérea, fotografia aérea de rastreamento, pesquisa e resgate,
monitoramento de gado, dentre outros. \cite{titounaLightweightSecurityTechnique2021} 

Esses sistemas possuem designs versáteis para atender diversas necessidades.
Esses dispositivos carregam dados cruciais para seus usuários, devido ao seu longo alcance,
para a realização de diversos trabalhos. \cite{khan}. 

Por serem operados de maneira autônoma e remota suas coordenadas são essenciais
para o sucesso de suas missões, para isso são usados sinais de Sistema Global de Navegação por Satélite (GNSS),
sendo o mais conhecido o \textit{Global Positioning System} (GPS), esses sinais podem ser criptografados ou não
\cite {lester}. Sinais militares possuem criptografia, enquanto os que são de uso civil não possuem mecanismos de proteção.
Com isso ocorre a extrema vulnerabilidade a diversos tipos de ataques, principalmente os conhecidos como \textit{Spoofing de GPS}.
\cite{srinivasansGPSSpoofingDetection2023}.

O tipo de ataque analisado nesse artigo é o \textit{spoofing}. 
Ele é caracterizado por uma técnica de ataque na qual o agressor faz uso de antenas terrestres emitindo sinais falsos
de localização \cite {Spoofing}.
Esse tipo de ação pode resultar em: danos materiais e o acesso indevido aos dados do usuário,
conforme relatado em diversos incidentes mundialmente.
\cite{g1drone2022a,g1drone2022b,bandnews2022,bemparana2022,veja2022,forbes2020,australianaviation2022}

Para garantir um vôo seguro do VANT é essencial ter medidas para detecção dos ataques de \textit{Spoofing de GPS}.
A identificação precisa desses ataques é extremamente importante, pois técnicas avançadas de \textit{spoofing}
podem causar disrupção no funcionamento do VANT. Portanto são necessários métodos robustos e adaptativos.
\cite{isleyenGPSSpoofingDetection2024}

Algoritmos com aprendizado de máquina tem se mostrado promissores, pois são capazes de analisar
grandes volumetrias de dados, identificando padrões e anomalias. Tais abordagens oferecem uma resposta robusta e adaptativa a essas ameaças contribuindo para as operações dos VANTs  \cite[tradução nossa]{isleyenGPSSpoofingDetection2024}.


\section{Delimitação e Escopo}
O \textit{dataset} que será utilizado como base de dados é o \textit{Mendeley Data} 
\cite{aissou2022dataset}, esses dados foram gerados de sinais GPS autênticos e 
contém aproximadamente 500,000 dados, que serão reduzidos a uma quantidade
considerável. Portanto, os resultados do trabalho ficarão limitados a este \textit{dataset}.

Nas próximas seções se estabelecerá a execução do trabalho. Na seção 3, faremos 
nossa proposta de mudança ao trabalho de \cite{Aissou2021}, discutindo a 
metodologia e modelos de aprendizado de máquina a
serem implementados. Logo, os resultados e 
discussões serão apresentados na seção 4. Finalmente, nossas conclusões sobre o 
trabalho serão expostos na seção 5.

\section{Metodologia (Proposta de mudança)}

Para alcançar o objetivo deste estudo, esta seção detalha a abordagem proposta para a classificação de possíveis ataques de \textit{spoofing} a sinais de GPS, avaliando o desempenho dos algoritmos: (i) Naive Bayes, (ii) K-NN, (iii) Extreme Random Forest, (iv) Random Forest e (v)  Gaussian Naive Bayes. 
A metodologia adotada e os passos seguidos foram inspirados no artigo de \citar o artigo\, com o objetivo de explorar os dados obtido. Serão analisados a acurácia, precisão e a quantidade de falsos negativos.

\bibliographystyle{sbc}
\bibliography{references}

\end{document}
